% =============================================================================
% FILE: chapters/chapter5.tex
% DESCRIPTION: Discussion
% =============================================================================

\chapter{DISCUSSION}

\section{Interpretation of Results}

GraviLens demonstrates that integrating hierarchical attention with physics-informed constraints significantly improves both accuracy and physical consistency.

\subsection{Superiority of Swin Transformer}

The hierarchical architecture naturally captures multi-scale lensing features, from fine arcs to global mass distributions, explaining the 5.1\% improvement over CNNs.

\subsection{Value of Physics-Informed Learning}

Direct embedding of SIE equations constrains the solution space, preventing non-physical predictions common in black-box models.

\section{Comparison with Existing Methods}

GraviLens outperforms:
\begin{itemize}
    \item Traditional CNN approaches (Hezaveh et al. 2017) in physical consistency
    \item LensPINN (Ojha et al. 2024) in computational efficiency
    \item Standard transformers in parameter accuracy
\end{itemize}

\section{Limitations}

\begin{itemize}
    \item Limited to SIE model approximation
    \item Trained on simulated data with domain gap to real observations
    \item Single-band imaging only
    \item Fixed 64×64 resolution
\end{itemize}

\section{Implications for Future Surveys}

The $\sim$100× speedup over traditional methods makes GraviLens suitable for processing millions of LSST and Euclid candidates.