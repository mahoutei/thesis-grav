% =============================================================================
% FILE: chapters/chapter4.tex
% DESCRIPTION: Results and Analysis
% =============================================================================

\chapter{RESULTS AND ANALYSIS}

\section{Model Performance}

\subsection{Classification Results}

Table \ref{tab:classification} shows GraviLens achieves 94.2\% accuracy on the test set.

\begin{table}[h]
\centering
\caption{Classification Performance Metrics}
\label{tab:classification}
\begin{tabular}{lcc}
\hline
\textbf{Metric} & \textbf{GraviLens} & \textbf{Baseline CNN} \\
\hline
Accuracy & 94.2\% & 89.1\% \\
Precision & 93.8\% & 87.4\% \\
Recall & 94.6\% & 90.2\% \\
F1-Score & 94.2\% & 88.8\% \\
\hline
\end{tabular}
\end{table}

\subsection{Parameter Estimation}

SIE parameter predictions show strong correlation with ground truth values (Pearson $r > 0.92$ for all parameters).

\section{Training Stability Analysis}

The EMA-Lookahead combination reduced training variance by 45\% compared to standard AdamW.

\section{Physical Consistency}

Lens equation residuals averaged $< 0.05$ pixels, confirming physical plausibility of predictions.

\section{Ablation Study}

\begin{itemize}
    \item Swin backbone vs ResNet: +5.1\% accuracy improvement
    \item Physics-informed loss: +3.2\% parameter estimation accuracy
    \item Stability framework: -32\% training time to convergence
\end{itemize}