% =============================================================================
% FILE: frontmatter/abstract.tex
% DESCRIPTION: Abstract of the thesis
% =============================================================================

\chapter*{ABSTRACT}
\addcontentsline{toc}{chapter}{ABSTRACT}
\thispagestyle{plain}

% Main Abstract Content
Gravitational lensing analysis is a critical tool for cosmology but poses a significant computational challenge for large-scale astronomical surveys. While deep learning offers a solution, standard Convolutional Neural Networks (CNNs) fail to capture the multi-scale physics of lensing and often produce physically inconsistent results. We introduce GraviLens, a novel framework integrating a Swin Transformer, a physics-informed encoder, and a stability framework to overcome these limitations. The Swin Transformer captures multi-scale lensing features, while the physics-informed encoder embeds the Singular Isothermal Ellipsoid (SIE) deflection model to ensure physical consistency. A dedicated stability framework guarantees robust training. Evaluated on real galaxy images, our model achieves superior performance in classification, marked by high accuracy and physical consistency. This work provides a powerful, reliable, and physically-grounded tool for automated lensing analysis, essential for future large-scale astronomical surveys.

\vspace{1cm}

\noindent\textbf{Keywords:} Gravitational lensing, deep learning, Swin Transformer, physics-informed neural networks, singular isothermal ellipsoid, dark matter, computer vision, astrophysics, machine learning

\cleardoublepage