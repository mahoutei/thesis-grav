% =============================================================================
% FILE: appendices/appendixB.tex
% DESCRIPTION: Code Implementation
% =============================================================================

\chapter{CODE IMPLEMENTATION DETAILS}

\section{Physics-Informed Layer Implementation}

\begin{verbatim}
class SIEDeflection(nn.Module):
    def __init__(self):
        super().__init__()
        
    def forward(self, params, coords):
        theta_E, q, phi = params
        x, y = coords
        
        # Rotate coordinates
        x_rot = x * cos(phi) + y * sin(phi)
        y_rot = -x * sin(phi) + y * cos(phi)
        
        # Compute deflection
        r = sqrt(q**2 * x_rot**2 + y_rot**2)
        alpha_x = theta_E * q**2 * x_rot / r
        alpha_y = theta_E * y_rot / r
        
        return alpha_x, alpha_y
\end{verbatim}

\section{Training Loop Pseudocode}

\begin{verbatim}
for epoch in range(num_epochs):
    for batch in dataloader:
        # Forward pass
        pred_class, pred_params = model(batch)
        
        # Compute losses
        loss_cls = BCELoss(pred_class, labels)
        loss_physics = SIELoss(pred_params, images)
        loss = loss_cls + lambda * loss_physics
        
        # Backward pass
        optimizer.zero_grad()
        loss.backward()
        clip_gradients(model.parameters())
        optimizer.step()
        
        # Update EMA
        update_ema(model, ema_model, decay=0.999)
\end{verbatim}

\section{Repository Information}

Full code available at: \url{https://github.com/[username]/gravilens}

% =============================================================================
% USAGE INSTRUCTIONS:
% =============================================================================
% 
% Save each chapter to:
%   chapters/chapter1.tex
%   chapters/chapter2.tex
%   chapters/chapter3.tex
%   chapters/chapter4.tex
%   chapters/chapter5.tex
%   chapters/chapter6.tex
%   appendices/appendixA.tex
%   appendices/appendixB.tex
%
% These are SHORT placeholders for testing compilation.
% Expand each section with your actual research content.
%
% =============================================================================